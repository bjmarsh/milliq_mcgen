\documentclass[12pt]{article}
\usepackage{hyperref}
\usepackage{amsmath}
%\usepackage{amsmath}


\title{Generation of decay of SM particles into millicharged particles}

\author{C. Campagnari, B. Marsh (UCSB)}

\begin{document}
\maketitle


\section{Introduction}
In this document we discuss the MC generation of decays and
the calculation of branching ratios for processes
of the type $A \to \zeta^+ \zeta^- X$ where
$A$ is a SM particle, $\zeta$ is a millicharged particle and $X$ are
other SM particles.  We start from $A \to e^+ e^- X$,
calculate the branching ratio for the decay into $\zeta$'s (if kinematically
allowed) and then discussed the methods to generate the actual decays.

We do not consider $Z \to \zeta \zeta$ since we expect that the Drell
Yan process should be dealt separately through an external MG model.
This is because the coupling of the $\zeta$ to the $Z$ are not given by
a simple rescaling of the SM $Zee$ vertex by $Q^2$, where $Q$ is the
charge of the $\zeta$ in units of $e$.

The generation of the SM $A$ particles is not discussed here.

\section{Processes}

The main processes for  $A \to e^+ e^- X$ at the LHC are:
\begin{itemize}
    \item $\pi^0 \to e^+ e^- \gamma$  (BR=1.17\%, Dalitz decay)
    \item $\eta \to e^+ e^- \gamma$   (BR=0.7\%, Dalitz decay)
    \item $\eta' \to e^+ e^- \gamma$   (BR=5e-4, Dalitz decay)
    \item $\omega \to \pi^0 e^+ e^-$  (BR=8e-4, Dalitz decay)
    \item $\eta' \to \omega e^+ e^-$ (BR=2e-4, Dalitz Decay)
    \item $\eta' \to \pi^+ \pi^- e^+ e^- \gamma$ (BR=2e-3, sort of Dalitz, skip for now)
    \item $V \to e^+ e^-$ ($V$ = onia, $\phi$, $\rho$, $\omega$)
\end{itemize}

Note that if we are only interested on $\zeta$ masses above 100 MeV,
Dalitz decays of $\pi^0$ and decays of $\eta'$ into $\omega$
do not contribute.
In all cases the BR for $A \to \zeta^+ \zeta^- X$ can be obtained
by rescaling $A \to e^+ e^- X$ by a factor of $Q^2$ times an
additional mass-dependent factor.  In general this factor consistes
of a phase space piece $\sqrt{1 - (2m_\zeta/m_A)^2}$ plus an additional
piece that arises from the matrix element.

\section{Branching Ratios}

\subsection{Dalitz BR}

The partial width for $A \to e^+ e^- \gamma$ can be written
as\cite{landsberg, bib:ulrik}:

\begin{equation}
\frac{d \Gamma}{d q^2} = \frac{2 \alpha}{3 \pi q^2}~~
(1+\frac{2m_e^2}{q^2})~~\sqrt{1 - \frac{4m_e^2}{q^2}}~~
(1 - \frac{q^2}{m_A^2})^3~~
|F(q^2)|^2 ~~
\Gamma(A \to \gamma \gamma) 
\label{dalitz1}
\end{equation}

\noindent where $m_A$ is the mass of $A$, $q^2$ is the mass-squared of the
$e^+e^-$ pair, 
and $F(q^2)$ is a form factor.
This form factor is such that $F(0)=1$ and for pions is usually
parametrized near $q^2=0$ as $F(q^2) = 1 + a \frac{q^2}{m_\pi^2}$ with
$a \approx 0.03$.  The form factor can also be estimated in the Vector
Dominance Model (VDM) as

\begin{equation}
  |F(q^2)|^2 = \frac{m^4_\rho + m^2_\rho \Gamma^2_\rho}{(m_\rho^2 - q^2)^2 + m^2_\rho \Gamma^2_\rho}
  \label{dalitz2}
\end{equation}

\noindent where $m_\rho$ and $\Gamma_\rho$ are the mass and width of the
$\rho$ meson.  The VDM model assumes that the decay proceeds through
$\pi^0 \to \gamma V^*, V^* \to e^+e^-$ and $V= \rho$ or $\omega$;
equation~\ref{dalitz2} neglects the difference between $\rho$ and $\omega$.

In the case of $A \to e^+ e^- X$, when $X$ is not $\gamma$, the
partial width can be written as 

\begin{equation}
\begin{split}
\frac{d \Gamma}{d q^2}  ~=~~~ & \frac{\alpha}{3 \pi q^2}~~
(1+\frac{2m_e^2}{q^2})~~\sqrt{1 - \frac{4m_e^2}{q^2}}~~ \cdot \\
& \left [ (1+\frac{q^2}{m_A^2-m_X^2})^2 - \frac{4
    m_A^2q^2}{(m_A^2-m_X^2)^2}\right ]^{3/2}~~
|F_{AX}(q^2)|^2 ~~
\Gamma(A \to X \gamma) 
\label{dalitz11}
\end{split}
\end{equation}

\noindent where $m_X$ is the mass of $X$, and the
transition
form factor $F_{AX}$ can also be approximated as in equation~\ref{dalitz2}.

For millicharged particles, Dalitz decays branching ratios 
can be obtained by integrating equation~\ref{dalitz1} or~\ref{dalitz11}
from 
$q^2=4m^2_\zeta$ to the kinematical limit, substituting $m_\zeta$ for 
$m_e$, and rescaling by $Q^2$.
Some numerical results, for $Q=1$ are given in Table~\ref{tab:dalitz1}.  Note the
sharp drop in branching ratios with mass, especially for the $\pi^0$.
The calculations
with the electron and muon masses are in good agreeemnt with the PDG.



\begin{table}
  \begin{center}
    { \small
  \begin{tabular}{|c|c|c|c|c|c|}
    \hline
    $m_\zeta$ (MeV) & $\pi^0 \to \zeta \zeta \gamma$ & $\eta \to \zeta \zeta \gamma$
    & $\eta' \to \zeta \zeta \gamma$ & $\eta' \to \zeta \zeta \omega$
    & $\omega \to \zeta \zeta \pi^0$ \\ \hline \hline
  0.511 (=$m_e$)  & 1.17 e-2 & 6.6 e-3 & 4.6 e-4 & 1.8 e-4 & 7.6 e-4 \\
  PDG for ee  & ($1.17 \pm 0.04$)e-2 & ($6.9 \pm 0.4$)e-4 &   &
  ($2.0 \pm 0.4$)e-4  & ($7.7 \pm 0.6$)e-4 \\ \hline
  10              & 2.8 e-3  & 2.9 e-3 & 2.5 e-4 & 5.7 e-5 & 3.7 e-4 \\
  30              & 3.5 e-4  & 1.6 e-3 & 1.8 e-4 & 1.7 e-5 & 2.3 e-4 \\
  50              & 1.2 e-5  & 1.0 e-3 & 1.4 e-4 & 4.3 e-6 & 1.6 e-4 \\
  60              & 2.7 e-7  & 8.2 e-4 & 1.3 e-4 & 1.7 e-6 & 1.4 e-4 \\
  90              &         & 4.3 e-4 & 1.0 e-4 &        & 9.2 e-5 \\ \hline
  105.7 (=$m_\mu$)&         & 3.0 e-4 & 9.2 e-5 &         & 7.4 e-5 \\
  PDG for $\mu\mu$ &      & ($3.1 \pm 0.4$) e-4 & ($1.1 \pm 0.3$) e-4 & &
                      ($1.3 \pm 0.2$) e-4 \\ \hline
  150             &        & 8.9 e-5  & 6.8 e-5 &        & 3.7 e-5 \\
  200             &        & 1.2 e-5  & 4.8 e-5 &        & 1.5 e-5 \\
  250             &        & 1.0 e-7 & 3.2 e-5 &         & 3.6 e-6 \\
  400             &        &        & 5.6 e-7 &         &        \\ \hline 
  \end{tabular}
  }
  \caption{\protect Branching ratios for different Dalitz decay modes as
    a function of $m_\zeta$ for $Q=1$ calculated
    based on equations~\ref{dalitz1},~\ref{dalitz2}, and~\ref{dalitz11}.
When possible we compare with
  the values from the 2019 PDG.}
\label{tab:dalitz1}
  \end{center}
\end{table}

\subsection{Vector meson branching ratios}

  At lowest order the SM decay rate for $V \to \ell \ell$ is given
  by the Van Royen-Weisskopf formula\cite{VR1,VR2}:

  \begin{equation}
    \Gamma(V \to \ell \ell) = 4 \pi \alpha^2 \frac{f_V^2}{m_V} Q_q^2
    (1-4x_\ell^2)^{1/2} (1+2x_\ell^2)
    \end{equation}

  \noindent where $f_V$ is the vector decay constant, $m_V$ is the
  vector mass, $Q_q$ is the charge of the quark that makes up the meson,
  $x_\ell = m_\ell/m_V$, and $m_\ell$ is the lepton mass.

  \noindent Thus the ratio of BR for $V \to \zeta \zeta$ to $V \to ee$ is
  given by

  \begin{equation} 
    \frac{\Gamma(V \to \zeta \zeta)}{\Gamma(V \to ee)} = Q^2
    \frac{(1-4x_\zeta^2)^{1/2} (1+2x_\zeta^2)}{(1-4x_\ell^2)^{1/2} (1+2x_\ell^2)}
    \label{V1}
    \end{equation}

  \noindent where $x_\zeta = m_\ell/m_V$, and $m_\zeta$ is the mass of $\zeta$.
  
  As a sanity check, we use equation~\ref{V1} and BR($\psi(2S) \to ee$)=7.93e-3
  to predict
  BR($\psi(2S) \to \tau \tau$) = 3.1e-3, in agreement with the PDG value
  of (3.1 $\pm$ 0.4)e-3.

\section{Generation of the decays}

We provide functions that take as input the lab frame 4-vector of either pseudoscalar
($P$) or vector ($V$) meson, and return the 4 vectors of the two $\zeta$'s from the decay.
We assume that the $V$'s are unpolarized.

\subsection{Generation of Dalitz decays}
The implementation goes as follows (this should also work for the Dalitz
decay of the $\omega$ as long as the $\omega$ is unpolarized):
\begin{itemize}
\item Rotate the 4-vector of $P$ from the lab frame into frame
  $S_1$ such that $P$ is traveling in the $z$-direction.
\item Boost along $z$ into frame $S_2$ where $P$ is at rest.
\item Pick a $q^2$ according to equation~\ref{dalitz1}.
\item Generate a decay $P \to X \gamma^*$ where $\gamma^*$ is
  a particle of $m^2 = q^2$.  The $\gamma^*$ direction is random
  in $\phi$ and random in $\cos \theta$.
\item Rotate the $\gamma^*$ 4-vector into a frame $S_3$ 
  such that the $\gamma^*$ is traveling in the $z$-direction.
\item Boost along $z$ into frame $S_4$ where $\gamma^*$ is at rest.
\item Generate a decay $\gamma^* \to \zeta^+ \zeta^-$ such that
  the angle $\phi$ of the $\zeta^+$ is random and
  $\cos \theta$ is picked according to~\cite{adlarson}
  \begin{equation}
    \frac{dN}{d \cos \theta} = 1 + \cos^2\theta + \frac{4 m^2_\zeta}{q^2} \sin^2\theta
    \label{dalitzAngles}
  \end{equation}
\item  Set the 3-vector of the $\zeta^-$ to be back-to-back with the $\zeta^+$.
\item Boost the 4-vectors of the $\zeta$'s from $S_4$ to $S_3$.
\item Rotate the 4-vectors of the $\zeta$'s from $S_3$ to $S_2$.
\item Boost the 4-vectors of the $\zeta$'s from $S_2$ to $S_1$.
\item Rotate the 4-vectors of the $\zeta$'s from $S_1$ to the lab frame.
\end{itemize}

\subsection{Generation of vector decays}

The procedure is the following:

\begin{itemize}
  \item Rotate the 4-vector of $V$ from the lab frame into frame
    $S_1$ such that $V$ is traveling in the $z$-direction.
  \item  Boost along $z$ into frame $S_2$ where $V$ is at rest.
  \item Generate a decay $V \to \zeta^+ \zeta^-$ such that
    both the angle $\phi$ and the $\cos \theta$ of the $\zeta^+$ are random.
  \item   Set the 3-vector of the $\zeta^-$ to be back-to-back with the $\zeta^+$.
  \item Boost the 4-vectors of the $\zeta$'s from $S_2$ to $S_1$.
\item Rotate the 4-vectors of the $\zeta$'s from $S_1$ to the lab frame.
\end{itemize}


\section{Code}
Code to calculate the branching ratios and to generate the decays
can be found at
\href{https://github.com/bjmarsh/milliq\_mcgen}
{https://github.com/bjmarsh/milliq\_mcgen}


\begin{thebibliography}{1}

\bibitem{landsberg}  L. G. Landsberg, Phys. Rep. 128, 301 (1985). 

\bibitem{bib:ulrik}
  See for example \href{http://cds.cern.ch/record/683210/files/soft-96-032.pdf}
{http://cds.cern.ch/record/683210/files/soft-96-032.pdf}.

\bibitem{VR1}
Aloni, D., Efrati, A., Grossman, Y. et al. J. High Energ. Phys. (2017) 2017: 19. 

\bibitem{VR2}
  R. Van Royen and V. F. Weisskopf, Nuovo Cim. A 50, 617 (1967) Erratum: [Nuovo Cim. A 51, 583 (1967)].

\bibitem{adlarson} P. Adlarson et al., Phys. Rev. C 95, 025202 (2007).
  
\end{thebibliography}
  
\end{document}
